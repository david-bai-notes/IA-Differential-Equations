\section{First-Order ODEs}
\begin{definition}
    An Ordinary Differential Equation (ODE) involves differentiable functions of $1$ variable, while Partial Differential Equations (PDEs) involve higher dimensional functions.\\
    The order of a differential equations is the highest order of derivative in the equation.
\end{definition}
\subsection{First-Order Linear ODEs}
An linear ODE is, obviously, a linear ODE.
\begin{example}
    $x^3y+y^\prime=0$ is a first-order linear ODE.
\end{example}
\subsubsection{Prelude: Exponential Function}
Consider the function $f(x)=a^x,a>1$, then $f^\prime(x)=a^x\lambda$ for some $\lambda>0$.
\begin{definition}
    We define $\exp(x)$ to be the solution to the differential equation $f^\prime=f$ with $f(0)=1$.
\end{definition}
By definition we have $(e^h-1)/h\to 1$ as $h\to 0$.
\begin{definition}
    One can show that $\exp$ is strictly increasing, therefore injective.
    So we can define $\ln$ to be the inverse function of $\exp$.
\end{definition}
So we have $\lambda=\ln a$.\\
The exponential function plays a central role in differential equations because it is the eigenfunction (i.e. a function that is only scaled under the operator) of the differential operator.\\
We obviously have $\mathrm de^{\lambda x}/\mathrm dx=\lambda e^{\lambda x}$, so it is indeed an eigenfunction.
Actually, all the eigenfunctions of the differential operator is of the form $Ce^{\lambda x}$ for some constant $C$ and $\lambda$.
So the behaviour of a differential equation can be somehow characterized by exponential functions.
\subsubsection{Rules of Linear ODEs}
Firstly, any homogeneous linear ODEs with constant coefficients (e.g. $af^\prime+bf=0, a,b\in\mathbb R$) have solutions of the form $Ce^{\lambda x}$ where $\lambda$ can be complex.\\
Secondly, for linear homogeneous ODEs, any contant multiple of a solution is also a solution.\\
Thirdly, an $n^{th}$ order linear ODE has only $n$ linearly independent solutions.\\
Lastly, an $n^{th}$ order ODE requires $n$ initial/boundary conditions to uniquely determine it.
\subsubsection{Forced (inhomogeneous) First Order ODEs with Constant Coefficient}
Case 1: Constant forcing.
For example, $5y'-3y=10$.
We will follow the following steps:\\
1. Find steady (equilibrium) solution where $y^\prime=0$.
In this case, it is $y=y_p=-10/3$.\\
2. Then, the general solution is in the form $y=y_p+y_c$ where $y_c$ is a complementary solution, any solution to the differential equation removing the inhomogeneous part.
in this case, $5y'-3y=0$.\\
3. Solve for $y_c$.
In this case, $y_c=Ae^{3x/5}$\\
4. Plug it back: $y=Ae^{3x/5}-10/3$\\
Here $A$ is any constant.
This way works since linearity of the equation granted that any two solutions to it must differ by a complementary solution.\\
Case 2: Eigenfunction forcing.\\
Suppose an isotope A decays into isotope B in a way that it is proportional to $a$, the number of nuclei of that isotope A.
While B decays into isotope C, its rate is proportional to $b$, the numbers of nuclei of isotope B.
So $\dot{a}=-k_aa\implies a=a_0e^{-k_at}$ for some constants $k_a,a_0$.
Now $\dot{b}=-k_bb+k_aa$ for some constant $k_b$, so
$$\dot{b}+k_bb=k_aa_0e^{-k_at}$$
This is an example where the forcing term is the eigenfunction of $\mathrm d/\mathrm dt$.
We can guess a particular solution
$$b_p=\frac{k_a}{k_b-k_a}e^{-k_at}$$
if $k_b\neq k_a$.
So the general solution is
$$b=\frac{k_a}{k_b-k_a}a_0e^{-k_at}+De^{-k_bt}$$
If $b(t)=0$, then
$$b=\frac{k_a}{k_b-k_a}a_0(e^{-k_at}-e^{-k_bt})$$
So
$$\frac{b}{a}=\frac{k_a}{k_b-k_a}(1-e^{(k_a-k_b)t})$$
we can solve it for $t$ and this solution allows us to date rocks etc by measuring the ratio of isotopes.
\subsubsection{First Order ODEs with Non-constant Coefficient}
The general form of these sort of equations are in the form $a(x)y^\prime+b(x)y=c(x)$ or the form $y^\prime+p(x)y=q(x)$ (given that $\forall x,a(x)\neq 0$).
We can solve it by integrating factor
$$\mu(x)=\exp\left(\int p\,\mathrm dx\right)$$
So
$$(\mu y)^\prime=\mu q\implies \mu y=\int \mu q\,\mathrm dx\implies y=\frac{1}{\mu}\int\mu q\,\mathrm dx$$
\subsection{Intermezzo: Discrete Equations}
Sometimes it is useful to consider functions evaluated at a discrete set of points, which could be useful to numerical integration and series solution.
\subsubsection{Numerical Integration}
One approximation to $\mathrm dy/\mathrm dx$ at $x=x_n$ can be written as
$$\left.\frac{\mathrm dy}{\mathrm dx}\right|_{y_n}\approx \frac{y_{n+1}-y_n}{h}$$
This is called the forward Euler approximation.
\begin{example}
    $5y'-3y=0$ again.
    Then it is approximately equal to the discrete equation
    $$5y_{n+1}-5y_n-3ny_n=0\implies y_{n+1}=\left(1+\frac{3h}{5}\right)y_n$$
    We can iterate it to approximate the solution given an initial value.
    This is the example of a recurrence relation.
    We can actually solve this by
    $$y_{n}=\left(1+\frac{3h}{5}\right)^ny_0$$
    Which can approximate the true solution pretty well if we let $h\to 0,n\to\infty$.
    Note that in this case, for finite value of $n$, $y_n$ is always less than the actual value of solution at that point.
\end{example}
\subsubsection{Series Solution}
A powerful way to solve DEs is to seek for solutions in the form of an infinite power series.
Let
$$y(x)=\sum_{n=0}^\infty a_nx^n$$
we can plug it in the differential equations to solve for $a_n$.
\begin{example}
    $5y'-3y=0$ yet again.
    Plug it in and we get
    $$5\sum_{n=0}^\infty (5(n+1)a_{n+1}-3a_n)x^{n}=0$$
    Hence
    $$5(n+1)a_n-3a_n=0\implies a_{n+1}=\frac{3a_n}{5n+5}\implies a_n=\left(\frac{3}{5}\right)^n\frac{1}{n!}a_0$$
    Thus
    $$y(x)=\sum_{n=0}^\infty a_nx^n=a_0\sum_{n=0}^\infty \left(\frac{3}{5}\right)^n\frac{1}{n!}x^n=a_0e^{3x/5}$$
    for some constant $a_0$.
\end{example}
\subsection{Non-linear First Order ODEs}
The general form of these sort of equations can be written in the following way:
$$Q(x,y)\frac{\mathrm dy}{\mathrm dx}+P(x,y)=0$$
We cannot guarantee that a nonlinear ODE can be solved in closed form.
But sometimes we can.
\subsubsection{Special Types of Non-linear First Order ODEs}
An ODE is said to be separable if it can be written in the form $q(y)\mathrm dy=p(x)\mathrm dx$, then we can solve for it by integrating both sides.\\
An ODE in the above form is called \textit{exact} if and only if $Q(x,y)\,\mathrm dy+P(x,y)\,\mathrm dx$ is an exact differential of some function $f(x,y)$, that is $\mathrm df=Q\,\mathrm dy+P\,\mathrm dx$.
If it is the case, our general form then give $\mathrm df=0$, so $f(x,y)=0$ is the solution.\\
To find it, we can make use of the multivariate chain rule to get
$$\frac{\partial f}{\partial x}+\frac{\partial f}{\partial y}\frac{\mathrm dy}{\mathrm dx}=0$$
$P=f_x,Q=f_y$, hence
$$\frac{\partial^2f}{\partial y\partial x}=\frac{\partial P}{\partial y},\frac{\partial^2f}{\partial x\partial y}=\frac{\partial Q}{\partial x}$$
So $P_y=Q_x$.
Conversely, if it is true in a simply connected domain, then $P\,\mathrm dx+Q\,\mathrm dy$ is an exact differential.
Therefore we can use it to test whether the given ODE is exact.
If so, we can find $f$ (hence a possibly implicit expression of $y$) back by integration.
\begin{example}
    We want to solve $6y(y-x)y'+(2x-3y^2)=0$.
    So $P=2x-3y^2,Q=6y(y-x)$, we can check that our preceding condition hold (i.e. $P_y=Q_x$), hence it is exact.\\
    To find $f(x,y)$, we notice
    $$
    \begin{cases}
        \left.\partial f/\partial x\right|_y=2x-3y^2\\
        \left.\partial f/\partial y\right|_x=6y(y-x)
    \end{cases}
    $$
    Integrating the first equation we have $f(x,y)=x^2-3xy^2+h(y)$ where $h$ is differentiable.
    Now plug it in the other equation we get $6y(y-x)=-6xy+h^\prime(y)$, thus $h^\prime(y)=6y^2\implies h(y)=2y^3-C$ where $C$ is a constant.
    So the general solution is
    $$x^2-3xy^2+2y^3=C$$
\end{example}
\subsubsection{Isoclines and Solution Curves}
Even if (most of the time) we cannot really solve the equation, we can still analyze its behaviour.
Now we consider an ODE of form $\dot{y}=f(y,t)$, each initial condition $y(t_0)=y_0$ will give a different solution curve (given existence).
\begin{example}
    Suppose $\dot{y}=t(1-y^2)$, of course this is separable and we can solve it, but without solving this, we can sketch solution curves as well.
\end{example}
\begin{definition}
    An isocline is a curve given by $\dot{y}$ being constant, that is, $t(1-y^2)=c$ for some constant $c$.
\end{definition}
Note that if $f(y,t)$ is single-valued (i.e. actually a function), then the solution curves cannot cross (unless as tangents of each other).
To sketch the solutions, we can first sketch all the Isoclines.
Note that along any isocline, the corresponding $\dot{y}$ is constant, so we can easily draw the vector field, so we could follow the directions with initial condition given to approximate the curve.\\
We can analysis the stability of a fixed point.
\begin{definition}
    A fixed point is a constant solution of $y$.
	That is, we have $\dot{y}=f(y,t)=0$.
\end{definition}
\begin{definition}
    A fixed point is called stable if the solution curve in a small neighbourhood of the fixed point converge to it.
\end{definition}
We now try to analyze the stability of fixed points using perturbation analysis.\\
Let $y=a$ be a fixed point of some DE $\dot{y}=f(y,t)$.
Consider a small perturbation of the fixed point $y=a+\epsilon(t)$, so $\dot{\epsilon}=f(a+\epsilon,t)=f(a,t)+\epsilon f_y(a,t)+O(\epsilon^2)$, so $\dot{\epsilon}\approx\epsilon f_y(a,t)$ which is linear.
The behaviour of $\epsilon$ obtained from this differential equation helps us to classify the fixed points.\\
If $\lim_{t\to\infty }\epsilon(t)=0$ then we call it a stable fixed point, if $\lim_{t\to\infty}\epsilon(t)=\pm\infty$, then we say it is unstable, otherwise we say it is neutral.
If $f_y(a,t)=0$, then we need higher order terms in that Taylor series in order to determine its behaviour.
\begin{example}
    $\dot{y}=t(1-y^2)$, so the fixed points are $y=\pm 1$.
    We have $f_y=-2ty$.\\
    For $y=1$, then $\dot{\epsilon}=-2t\epsilon\implies \epsilon_{0}e^{-t^2}$, so $\epsilon\to 0$ when $t\to\infty$.
    So it is stable.\\
    For $y=-1$, then $\dot{\epsilon}=2t\epsilon\implies \epsilon_{0}e^{t^2}$, so $\epsilon\to\pm\infty$ (for $\epsilon_0\neq 0$) when $t\to\infty$.
    So it is unstable.
\end{example}
\subsubsection{Autonomous DEs}
\begin{definition}
    A DE is called autonomous if $f$ does not depend on $t$, that is, it is of the form $\dot{y}=f(y)$.
\end{definition}
So in this case, we apply the perturbation analysis to get $\dot{\epsilon}=f^\prime(a)\epsilon$, thus $\epsilon=\epsilon_0e^{kt}$ where $k=f^\prime(a)$, so if $\epsilon_0\neq 0$,
$$\epsilon\to
\begin{cases}
    0\text{, if $f^\prime(a)<0$, so it is stable}\\
    \pm\infty\text{, if $f^\prime(a)>0$, so it is unstable}\\
    \text{Others, if $f^\prime(a)=0$, so it is neutral}
\end{cases}$$
\subsubsection{Phase Portraits}
Another way to analyze the behaviour of a given DE is from a geometric perspective represented by something called phase portrait.
\begin{example}
    Chemical kinetics is an important example why it is useful.
    Consider neutralization reaction ${\rm NaOH+HCl=H_2O+NaCl}$.
    Let $a(t),b(t)$ be the numbers of NaOH and HCl molecules at times $t$, and $c(t)$ be the number of ${\rm H_2O}$ which is equal to that of NaCl.\\
    So the initial condition is $a(0)=a_0,b(0)=b_0$, and the model is $\dot{c}=\lambda ab$, also $a=a_0-c,b=b_0-c$, so
    $$\frac{\mathrm dc}{\mathrm dt}=\lambda(a_0-c)(b_0-c)$$
    which is autonomous, so $a_0,b_0$ are the fixed points of the DE.
    We can sketch the graph of $\dot{c}$ against $c$, which is called the 2D phase protrait of the DE, where we can analyze the attraction vectors near the phase protrait by the trends of it to decide the stability.
\end{example}
\begin{example}\label{logistic_cont}
    Let $y(t)$ be the population at time $T$ and birth rate would be $\alpha y$ for some $\alpha$ and the death rate be $\beta y$ for some $\beta$.\\
    Case 1: Linear model.
    So $\dot{y}=\alpha y-\beta y\implies y=y_0e^{(\alpha -\beta)t}$, so $y\to\infty$ if $\alpha>\beta$.\\
    Case 2: Nonlinear model.
    So $\dot{y}=(\alpha-\beta)y-\gamma y^2$.
    the parameter $\gamma$ here could be due to the death due to overcrowdedness, etc.
    Equivalently, we can write it as $\dot{y}=ry(1-y/\lambda)$ where $r=\alpha-\beta,\lambda=(\alpha-\beta)/\lambda$, so we can sketch the phase protrait again.
    Note that the fixed points are $0$ and $\lambda$, the former is unstable but the latter is stable.
\end{example}
\subsubsection{Fixed Points in Discrete Equations}
We can introduce fixed points in discrete equations as well.
Consider a first order discrete equation (aka difference equation) of the form $x_{n+1}=f(x_n)$.
\begin{definition}
    The fixed point is a discrete equation is a fixed point of the function $f$.
\end{definition}
We can analyze its stability again by perturbation analysis.
Let $x_f$ be a fixed point and we perturb it by a small $\epsilon$, so we can expand $f$ in terms of Taylor series
$$f(x_f+\epsilon)=x_f+\epsilon f^\prime(x_f)+O(\epsilon^2),\epsilon\to0$$
So if $x_n=x_f+\epsilon, x_{n+1}=x_f+\epsilon f^\prime(x_f)$.
So $x_f$ is stable if $|f^\prime(x_f)|<1$, neutral if $|f^\prime(x_f)|=1$, unstable if $|f^\prime(x_f)|>1$.
\begin{example}[Discrete Logistic Equation]
    Some populations are born in distinct generations (e.g. lambs born in spring).
    A nonlinear model of this is
    $$\frac{x_{n+1}-x_n}{\Delta t}=\lambda x_n-\gamma x_n^2$$
    This is the discrete version of Example \ref{logistic_cont}, so
    $$x_{n+1}=(1+\lambda\Delta t)x_n-\gamma\Delta tx_n^2$$
    We can write alternatively $x_{n+1}=rx_n(1-x_n)=:f(x_n)$
    The function $f$ is called the logistic map.\\
    Note that the fixed point occurs at $0$ or $1-1/r$.
    We analyze its stability by perturbation.
    At $0$, $f^\prime=r$, so it is stable if $0<r<1$ and unstable if $r>1$.
    At $1-1/r$, $f^\prime=2-r$.
    We assume $r<0$ or $r>1$ since the other cases are not physical.
    \footnote{It's an applied course, so what the hell.}
    Then it is stable for $1<r<3$ and unstable for $r>3$.
\end{example}
