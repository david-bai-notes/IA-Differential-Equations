\section{Order of Magnitude and Taylor's Theorem}
\subsection{Asymptopic Behaviour}
When we want to analyze the difference between functions, apart from their analytical properties, we would also be interested in the difference between their magnitudes.
The following notion arrives to serve this purpose
\begin{definition}[Little-$o$ notation]
    Let $f, g$ be real functions and $x_0\in\bar{\mathbb R}=\mathbb R\cup\{\pm\infty\}$, we say that $f(x)=o(g(x))$ as $x\to x_0$ if
    $$\lim_{x\to x_0}\frac{f(x)}{g(x)}=0$$
\end{definition}
\begin{definition}[Big-$O$ notation]
    Let $f, g$ be real functions and $x_0\in\mathbb R$, we say that $f(x)=O(g(x))$ as $x\to x_0$ if $\exists\delta,M>0$ such that
    $$|x-x_0|<\delta\implies |f(x)|\le M|g(x)|$$
    We say $f(x)=O(g(x))$ as $x\to\infty$ if $\exists x_1,M>0$ such that
    $$x>x_1\implies |f(x)|\le M|g(x)|$$
    We say $f(x)=O(g(x))$ as $x\to-\infty$ if $\exists x_1<0,M>0$ such that
    $$x<x_1\implies |f(x)|\le M|g(x)|$$
\end{definition}
\begin{remark}
    Remember that the equality sign of $f(x)=o(g(x))$ or $f(x)=O(g(x))$ is not really the usual equality sign we use.
    It is more like $f(x)\in o(g(x))$ or $f(x)\in O(g(x))$, meaning that $f$ is one of those functions having this property on its magnitude.
    The reason why we use the equality sign here is that we sometimes use the notations to denote \textit{some} functions having this property, which we do not (need to) know any detail except its magnitude.
\end{remark}
\begin{example}
    We have $x^2=o(x)$ as $x\to0$ and $x^2=O(x)$ as $x\to0$. In fact, whenever $f(x)=o(g(x))$ as $x\to x_0$ we have $f(x)=O(g(x))$ as $x\to x_0$ as well. The proof again is just checking definitions.
\end{example}
Now, one of the most significant usage of the measurement of magnitude is that we can use it to approximate the rest of the terms in a series.
For example, if we take the series $1+x+x^2+x^3+\ldots$, we can replace it by $1+x+x^2+x^3+o(x^3)$, in which way one can include information about the magnitude of the error term of the series.
\subsection{Taylor Series}
The idea of the Taylor series is to locally approximate a smooth enough function by polynomials.
Surely, most functions have much worse analytical properties then polynomials, making it slightly problematic to analyze some of their properties.
Taylor series provides a solution.
Basically, we start by assuming a local approximation of the function by a polynomial.
Say $f(x)\approx a_0+a_1x+a_2x^2+\cdots+a_nx^n$, then, by differentiating both sides recursively, we immediately have $a_n=f^{(n)}(0)/n!$.
The polynomial, which we will call $P_{n,0}(x)$, is called the Taylor polynomial.
In general, if we shift the polynomial by $x_0$, we have the general form:
\begin{definition}
    The Taylor polynomial $P_{n,x_0}(x)$ of a function $f$ around a point $x_0$ is the polynomial
    $$\sum_{k=0}^{n}\frac{f^{(k)}(x_0)(x-x_0)^k}{k!}$$
\end{definition}
Now the big question is, we have (quite vaguely) obtained the form of the sequence of polynomials that looks as if it can approximate $f$ as $n$ is big enough.
But does it?
Obviously unless $f$ is a polynomial as well it has no chance that the polynomial will be equal to $f$, but what can we say about the magnitude of the error term?
Taylor's theorem saves the day.
\begin{theorem}[Taylor's theorem]
    Write $h=x-x_0$.
	Provided that $f^{(n+1)}$ exists, then
    $$E_{n,x_0}(x)=f(x)-P_{n,x_0}(x)=O(h^{n+1})$$
    as $h\to 0$.
\end{theorem}
Actually $E_{n,x_0}=o(h^n)$ as $h\to 0$ as well but the big-$O$ here is stronger.
\subsection{L'Hopital's Rule}
\begin{theorem}
    If $f(x)$ and $g(x)$ are both differentiable at $x=x_0\in\bar{\mathbb R}$, and that $f,g$ are both continuous at $x_0$ and $f(x_0)=g(x_0)=0$, then
    $$\lim_{x\to x_0}\frac{f(x)}{g(x)}=\lim_{x\to x_0}\frac{f^\prime(x)}{g^\prime(x)}$$
\end{theorem}
\begin{proof}
    The little-$o$ notations below are taken as $x\to x_0$.
    $$f(x)=f(x_0)+(x-x_0)f^\prime(x_0)+o(x-x_0), g(x)=g(x_0)+(x-x_0)g^\prime(x_0)+o(x-x_0)$$
    by Taylor's theorem.
    $$\frac{f(x)}{g(x)}=\frac{f^\prime(x_0)+o(x-x_0)/(x-x_0)}{g^\prime(x_0)+o(x-x_0)/(x-x_0)}\to \frac{f^\prime(x_0)}{g^\prime(x_0)}$$
    as $x\to x_0$.
\end{proof}
Note that we can use L'Hopital's rule recursively given that the conditions still hold.
