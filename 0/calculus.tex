\section{Basic Calculus}
\begin{definition}
    Let $f:U\to\mathbb R$ where $U\subset\mathbb R$, then if $\forall\epsilon>0,\exists\delta>0$ such that
    $$|x-x_0|<\delta\implies|f(x)-A|<\epsilon$$
    for some $x_0,A\in\mathbb R$, we say
    $$\lim_{x\to x_0}f(x)=A$$
\end{definition}
\begin{theorem}
    Assume that
    $$\lim_{x\to x_0}f(x)=A, \lim_{x\to x_0}f(x)=B$$
    Then the followings hold:
    $$\lim_{x\to x_0}cf(x)=cA$$
    $$\lim_{x\to x_0}f(x)+g(x)=A+B$$
    $$\lim_{x\to x_0}f(x)g(x)=AB$$
    $$B\neq 0\implies\lim_{x\to x_0}f(x)/g(x)=A/B$$
\end{theorem}
\begin{proof}
    Check definitions.
\end{proof}
\begin{definition}[One-sided limits]
    If $\forall\epsilon>0,\exists\delta>0$ such that $0<x-x_0<\delta\implies|f(x)-A|<\epsilon$
    for some $x_0,A\in\mathbb R$, we say
    $$\lim_{x\to x_0^+}f(x)=A$$
    Simialrly, if it is $0<x_0-x<\delta\implies|f(x)-A|<\epsilon$, then
    $$\lim_{x\to x_0^-}f(x)=A$$
\end{definition}
\begin{definition}
    If the limit
    $$\lim_{h\to0}\frac{f(x+h)-f(x)}{h}$$
    exists for all $x$ in a given domain, we say that $f$ is differentiable in this domain and its derivative is
    $$f^\prime(x)=\frac{\mathrm df}{\mathrm dx}=\lim_{h\to0}\frac{f(x+h)-f(x)}{h}$$
\end{definition}
For sufficiently smooth functions, we can differentiate it recursively.
We denote the $n^{th}$ derivative of $f$ as
$$\frac{d^nf}{dx^n}\text{ or }f^{(n)}(x)$$
Several immediate facts are available from here:
\begin{theorem}
    1. The differential operator is linear.\\
    2. (Chain Rule) Suppose both $F$ and $g$ are differentiable and $f(x)=F(g(x))$,
    then $f$ is differentiable and $f^\prime(x)=F^\prime(g(x))g^\prime(x)$.\\
    3. (Leibniz's Rule) Suppose both $u$ and $v$ are differentiable and $f(x)=u(x)v(x)$, then
    $$f^{(n)}(x)=\sum_{k=0}^n\binom{n}{k}u^{(k)}(x)v^{(n-k)}(x)$$
    in particular $f^\prime(x)=u(x)v^\prime(x)+u^\prime(x)v(x)$.
\end{theorem}
\begin{proof}
    Trivial.
\end{proof}