\section{Multivariate Calculus}
In many real world applications, functions we might be interested in can involve more than one independent variable.
\begin{example}
    Waves along a string.
    Let $f(x,t)$ be the displacement where $x$ be the position and $t$ be the time.
    The shape of the wave (represented by $f(x,t_0)$ where $t_0$ is fixed) is dependent on time.
\end{example}
How do we define the derivative when a function depends on more than one variable?
Suppose that $f(x,y)$ is the elevation of the terrain at the specific location $x,y$.
We can draw a contour map of its projection on a plane.
If we are interested in the steepness at a point $a$ on the surface, one should note that different trails going though $a$ may have different steepness.
So the general point is that the slope of a function at a given point depends on the direction.
\subsection{Partial Derivative}
We want to find the derivative of a multivariate function with respect to one variable while keeping others fixed.
\begin{definition}
    Mathematically speaking, we define partial derivative $f(x,y)$ with respect to $x$ fixing $y$ is the limit
    $$\left.\frac{\partial f}{\partial x}\right|_y=\lim_{h\to0}\frac{f(x+h,y)-f(x,y)}{h}$$
    provided that it exists.
    We can define the partial derivative with respect to $y$ fixing $x$ similarly.
\end{definition}
\begin{example}
    Let $F(x,y)=x^2+y^3+e^{xy^2}$, so
    $$\left.\frac{\partial f}{\partial x}\right|_y=2x+y^2e^{xy^2},\left.\frac{\partial f}{\partial y}\right|_x=3y^2+2xye^{xy^2}$$
\end{example}
We can do partial derivatives recursively as well.
$$\left.\frac{\partial^2 f}{\partial x^2}\right|_y=2+y^4e^{xy^2}$$
Now we can define cross-derivative as well,
$$\left.\frac{\partial}{\partial y}\left(\left.\frac{\partial f}{\partial x}\right|_y\right)\right|_x=2ye^{xy^2}+2xy^3e^{xy^2}$$
Since the notation is cumbersome, we sometimes omit the symbol $|_y$.\\
There is a symmetry involved in mixed partial derivatives.
By that we mean
$$\frac{\partial^2 f}{\partial x\partial y}=\frac{\partial^2 f}{\partial y\partial x}$$
given that all these partial derivatives exist.
Some properties are required for this equality to hold, but they are out of the scope of this course.\\
On higher dimensions (where we can define partial derivatives analogously), for example $f(x,y,z)$, when we sometimes say
$$\left.\frac{\partial f}{\partial x}\right|_y$$
its value would depend on the path it takes in the $x-z$ plane.\\
We sometimes use the shorthand notation $f_x,f_{xy},f_{xx}$ for the partial derivatives.
\subsection{The Chain Rule on Higher Dimensions}
Consider $f(x(t),y(t))$, we first want to have the concept of a differential of a function.
$$\delta f=f(x+\delta x, y+\delta y)-f(x,y)$$
So we have
$$\delta f=f(x+\delta x, y+\delta y)-f(x+\delta x,y)+f(x+\delta x,y)-f(x,y)$$
When $y$ is held constant, $f(x+\delta x,y)=f(x,y)+\delta x(\partial f/\partial x)(x,y)+o(\delta x)$ as $\delta x\to0$.
Similarly $f(x+\delta x,y+\delta y)=f(x,y)+\delta y(\partial f/\partial y)(x+\delta x,y)+o(\delta y)$ as $\delta y\to0$.
Therefore as $\delta x\to 0,\delta y\to0$
\begin{align*}
    \delta f
    &=f(x+\delta x,y)+\delta y\frac{\partial f}{\partial y}(x+\delta x,y)+o(\delta y)-f(x+\delta x,y)\\
    &\quad+f(x,y)+\delta x\frac{\partial f}{\partial x}(x,y)+o(\delta x)-f(x,y)\\
    &=\delta y\frac{\partial f}{\partial y}(x+\delta x,y)+\delta x\frac{\partial f}{\partial x}(x,y)+o(\delta x)+o(\delta y)\\
    &=\delta y\frac{\partial f}{\partial y}(x,y)+\delta x\frac{\partial f}{\partial x}(x,y)\\
    &\quad+\delta x\delta y\frac{\partial}{\partial x}\frac{\partial f}{\partial y}(x,y)+o(\delta x)+o(\delta y)+o(\delta x\delta y)
\end{align*}
Note that $o(\delta x\delta y)+o(\delta x)=o(\delta x)$ as $\delta y\to0$, so we can ignore that term.\\
We let $\delta x,\delta y\to 0$, so
\begin{align*}
    \delta f&=\delta y\frac{\partial f}{\partial y}+\delta x\frac{\partial f}{\partial x}+\delta x\delta y\frac{\partial}{\partial x}\frac{\partial f}{\partial y}+o(\delta x)+o(\delta y)\\
    \mathrm df&=\frac{\partial f}{\partial y}\mathrm dy+\frac{\partial f}{\partial x}\mathrm dx
\end{align*}
This is called the chain rule in differential form.
We can obtain the chain rule by dividing by another differential $\mathrm dt$ before applying the limit.
$$\frac{\mathrm df}{\mathrm dt}=\frac{\partial f}{\partial y}\frac{\mathrm dy}{\mathrm dt}+\frac{\partial f}{\partial x}\frac{\mathrm dx}{\mathrm dt}$$
This is called the multivariate chain rule.\\
Suppose $f(x,y(x))$, then
$$\frac{\mathrm df}{\mathrm dx}=\frac{\partial f}{\partial x}+\frac{\partial f}{\partial y}\frac{\mathrm dy}{\mathrm dx}$$
We can integrate it back
$$\int\mathrm df=\int\frac{\partial f}{\partial x}\,\mathrm dx+\int\frac{\partial f}{\partial y}\,\mathrm dy$$
Note that we need to integrate the above equation along a given path, but if the function is nice enough, only the endpoints matter.
\begin{example}
    We choose the paths
    $$(x_1,y_1)\to(x_2,y_1)\to(x_2,y_2)$$
    and
    $$(x_1,y_1)\to(x_1,y_2)\to(x_2,y_2)$$
    then
    $$f(x_2,y_2)-f(x_1,y_1)=\int_{x_1}^{x_2}\frac{\partial f}{\partial x}(x,y_1)\,\mathrm dx+\int_{y_1}^{y_2}\frac{\partial f}{\partial y}(x_2,y)\,\mathrm dy$$
    $$=\int_{x_1}^{x_2}\frac{\partial f}{\partial x}(x,y_2)\,\mathrm dx+\int_{y_1}^{y_2}\frac{\partial f}{\partial y}(x_1,y)\,\mathrm dy$$
\end{example}
An application of the multivariate chain rule is the change of variables.
It is often useful to write a differential equation in a different coordinate system before solving it.
To do this, we need to transform the derivatives from one to the other.
\begin{example}
    We try to transform from Cartesian to polar coordinate, so $x=r\cos\theta,y=r\sin\theta$, so we can write $f(x(r,\theta),y(r,\theta))$, so
    $$\left.\frac{\partial f}{\partial r}\right|_\theta=\left.\frac{\partial f}{\partial x}\right|_y\left.\frac{\partial x}{\partial r}\right|_\theta+\left.\frac{\partial f}{\partial y}\right|_x\left.\frac{\partial y}{\partial r}\right|_\theta=\left.\frac{\partial f}{\partial x}\right|_y\cos\theta+\left.\frac{\partial f}{\partial y}\right|_x\sin\theta$$
\end{example}
We can also apply it to implicit differentiation.
Consider $f(x,y,z)=c$ where $c$ is a constant.
It implicitly defined $z(x,y),x(y,z),y(z,x)$.
For example, we can take $xy+y^2z+z^5=1$, so $x=(1-z^5-y^2z)/y$ and we can find $y$ by quadratic formula, but we cannot do it easily with $z$ since it's quintic.\\
However, we can find the derivative $\partial z/\partial x$ fixing $y$ by observing
$$0=\left.\frac{\partial f}{\partial x}\right|_y=y+y^2\left.\frac{\partial z}{\partial x}\right|_y+5z^4\left.\frac{\partial z}{\partial x}\right|_y$$
which we can solve for the desired derivative.\\
Now consider $f(x,y,z(x,y))=0$,
$$\mathrm df=\left.\frac{\partial f}{\partial x}\right|_{y,z}\mathrm dx+\left.\frac{\partial f}{\partial y}\right|_{x,z}\mathrm dy+\left.\frac{\partial f}{\partial z}\right|_{x,y}\mathrm dz$$
We want the derivative of $z$ wrt $x$ with $y$ fixed, so
$$\left.\frac{\partial f}{\partial x}\right|_y=\left.\frac{\partial f}{\partial x}\right|_{y,z}+\left.\frac{\partial f}{\partial z}\right|_{x,y}\left.\frac{\partial z}{\partial x}\right|_y$$
In fact, $(\partial f/\partial x)|_y=0$.\\
The chain rule allows us to differentiate an integral.
Consider  function $f(x,c)$ where each value of $c$ gives a different function $f$.
We want to find the (partial) derivative of the integral
$$\int_0^bf(x,c)\,\mathrm dx$$
Then
\begin{align*}
    \left.\frac{\partial I(b,c)}{\partial b}\right|_c
    &=\lim_{h\to 0}\frac{1}{h}\int_b^{b+h}f(x,c)\,\mathrm dx\\
    &=f(b,c)
\end{align*}
Similarly,
\begin{align*}
    \left.\frac{\partial I(b,c)}{\partial c}\right|_b
    &=\lim_{h\to 0}\frac{1}{h}\int_0^bf(x,c+h)-f(x,c)\,\mathrm dx\\
    &=\int_0^b\left.\frac{\partial I(x,c)}{\partial c}\right|_x\,\mathrm dx
\end{align*}
Suppose that $b,c$ depends on $t$, so $I(b(t),c(t))$, so
\begin{align*}
    \frac{\mathrm dI}{\mathrm dt}
    &=\frac{\partial I}{\partial b}\frac{\mathrm db}{\mathrm dt}+\frac{\partial I}{\partial c}\frac{\mathrm dc}{\mathrm dt}\\
    &=f(b,c)\dot{b}+\dot{c}\int_0^b\left.\frac{\partial I(x,c)}{\partial c}\right|_x\,\mathrm dx
\end{align*}
In general,
$$\frac{\mathrm d}{\mathrm dt}\int_{a(t)}^{b(t)}f(x,c(t))\,\mathrm dx=\dot{c}\int_{a(t)}^{b(t)}\frac{\partial f}{\partial c}\,\mathrm dx+f(b,c)\dot{b}-f(a,c)\dot{a}$$
Note that the reciprocal rule also apply.
The same rule applies to partial derivatives given that the same parameter is kept constant.
$$\left.\frac{\partial r}{\partial x}\right|_y=\frac{1}{\partial x/\partial r|_y}$$