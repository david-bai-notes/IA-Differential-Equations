\section{Fundamental Theorem of Calculus}
\subsection{Integration}
Assume below that all functions under consideration are nice enough to let the integral exist.\\
Consider the following sum
$$\sum_{n=0}^{N-1}f(x_n)\Delta x$$
where $\Delta x=(b-a)/N, x_n=a+n\Delta x$.\\
The question is, how close is the finite sum above to the area under the curve of $f$, when $N$ is large?
How big is the difference between the difference between the area and the discrete area in the sum?
\begin{theorem}[Mean-vaue Theorem on Definite Integrals]
    For a continuous function $f(x)$,
    $$\int_{x_n}^{x_n+1}f(x)\,\mathrm dx=f(x_c)(x_{n+1}-x_n)$$
    for some $x_c\in(x_n,x_{n+1})$.
\end{theorem}
Expand $f(x)$ about $x\to x_n$ and evaluate it at $x_c$.
As $\Delta x\to 0$,
$$f(x_c)=f(x_n)+O(x_c-x_n)=f(x_n)+O(x_{n+1}-x_n)$$
as $|x_c-x_n|<|x_{n+1}-x_n|$.
Hence by the mean-value theorem:
\begin{align*}
    \int_{x_n}^{x_n+1}f(x)\,\mathrm dx
    &=f(x_n)(x_{n+1}-x_n)+O(x_{n+1}-x_n)(x_{n+1}-x_n)\\
    &=\Delta xf(x_n)+O(\Delta x^2)
\end{align*}
So $\epsilon=O(\Delta x^2)$.
It follows that
$$\int_a^bf(x)\,\mathrm dx=\lim_{N\to\infty}\sum_{n=0}^{N-1}f(x_n)\Delta x+\epsilon_n$$
By our bounds above, the error terms $\sum_n\epsilon_n=O(N\Delta x^2)=O((b-a)^2/N)$ vanish as $N\to\infty$.
Therefore,
$$\int_a^bf(x)\,\mathrm dx=\lim_{N\to\infty}\sum_{n=0}^{N-1}f(x_n)\Delta x$$
\begin{theorem}[Fundamental Theorem of Calculus]
    Let
    $$F(x)=\int_a^xf(t)\,\mathrm dt$$
    then $\mathrm dF/\mathrm dx=f(x)$.
\end{theorem}
\begin{proof}
    We try to evaluate the derivative of $F$,
    \begin{align*}
        \frac{\mathrm dF}{\mathrm dx}
        &=\lim_{h\to0}\frac{1}{h}\int_{x}^{x+h}f(t)\,\mathrm dt\\
        &=\lim_{h\to0}\frac{1}{h}(f(x)h+O(h^2))\\
        &=f(x)
    \end{align*}
    So $\mathrm dF/\mathrm dx=f(x)$.
\end{proof}
\begin{corollary}
    $$\frac{\mathrm d}{\mathrm dx}\int_x^bf(t)\,\mathrm dt=-f(x)$$
\end{corollary}
\begin{corollary}
    $$\int_a^{g(x)}f(t)\,\mathrm dt=f(g(x))g^\prime(x)$$
\end{corollary}
\begin{corollary}
    Let
    $$F(x)=\int f(x)\,\mathrm dx$$
    be the indefinite integral (or anti-derivative) of $f$, then
    $$\int_a^bf(t)\,\mathrm dt=F(b)-F(a)$$
\end{corollary}
\subsection{Some Integration Techniques}
The first one is integration by substitution.
\begin{example}
    $$\int\frac{1-2x}{\sqrt{x-x^2}}\,\mathrm dx=\int\frac{\mathrm du}{\sqrt u}=2\sqrt{u}+C$$
\end{example}
Trigonometric substitution
\begin{example}
    If we see something like $\sqrt{a^2-x^2}$, then we can substitute $x=a\sin\theta$.\\
    If we see something like $x^2+a^2$, we can use $x=a\tan\theta$.\\
    If we see $\sqrt{x^2-a^2}$, we can use $x=a\cosh\theta$ or $x=a\sec\theta$.\\
    If we see $\sqrt{x^2+a^2}$, we can use $x=a\sinh\theta$ or $x=a\tan\theta$.\\
    If we see $a^2-x^2$, we can use $x=a\tanh\theta$
\end{example}
And, of course, we have integration by part:
$$\int uv^\prime=uv-\int u^\prime v$$
from product rule.